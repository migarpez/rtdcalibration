\section{Conclusions}
\label{conclusions}

\noindent The DUNE experiment will require the construction and operation of the largest cryostats ever used in a particle physics experiment. This makes the continuous measurement of temperature gradients in liquid argon crucial for monitoring the stability of the cryogenics system and for detector calibration. R\&D on the calibration of RTD probes started in 2017, leading to promising results for sensors installed in the DUNE prototype at CERN.

The first setup proved the viability of the method, obtaining a calibration error of 1.7 mK. A key component was the readout electronics, with an intrinsic resolution better than 0.5 mK in the comparison between two channels. The mechanics was also crucial, with several insulation layers consisting of independent concentric volumes, ensuring minimal convection in the inner volume. Sensors were contained in an aluminium capsule, enabling slow cool-down and warm-up processes, found to be fundamental to guarantee the integrity of the sensors and to minimize the effect of ageing.

The calibration system was later enhanced to accommodate the large-scale calibration required for the DUNE detectors. The capacity of the inner capsule was increased from 4 to 14 sensors, while improvements were made to the insulation and symmetry of the system to minimize temperature differences between sensors. The new system has slightly worst repeatability for sensors in the same set, but reduces the statistical and systematic errors associated to the calibration tree, needed to relate any two sensors in different calibration sets. Calibration with the new setup has achieved a precision in the range of 1.6-3.0 mK, substantially better than the 5 mK DUNE requirement.

Another difference between the new and old calibrations is the use of a different cryogenic liquid. While DUNE will use LAr, LN2 is cheaper, simplifying the process of massive calibration for DUNE detectors. The 10 K difference between those liquids has a minimal effect on the calibration constants.

A comparison of the four calibration campaigns has provided valuable insights into aging effects, with no evidence of RTD aging observed over a five-year period. This highlights the stability and reliability of the PT-102-based system.