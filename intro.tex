\section{Introduction}
\label{sec:introduction}

\noindent Precise temperature sensing is essential in large cryogenic detectors, where the circulation and purification of cryogenic liquids can have a major impact on overall detector performance. In these systems, uncontrolled temperature differences can modify the cryostat dynamics, ultimately affecting the physics sensitivity of the experiment. In this work, we present the calibration procedure of a novel temperature monitoring system developed for a prototype of the future DUNE experiment. 

DUNE, which stands for Deep Underground Neutrino Experiment~\cite{dune_tdr1}, is expected to begin taking data towards the end of the decade. It aims at performing comprehensive neutrino oscillation analyses, ---broadly exploring the Charge-Parity violation parameters phase-space and resolving the neutrino-mass hierarchy problem \cite{bib:dune_osc}. Its vast physics program also includes searches for hypothetical proton decay channels \cite{bib:dune_tdr2}, multi-messenger astronomy from supernovae and neutrino bursts \cite{bib:dune_supernova}, and explorations of Beyond the Standard Model (BSM) physics \cite{bib:dune_bsm}. With the most powerful neutrino beam ever built, produced at Fermilab, the experiment follows the long-baseline neutrino oscillations approach, with two detectors. The Near Detector, also at Fermilab, will characterize the unoscillated neutrino beam. The Far Detector (FD), located in the Sandford Underground Research Facility (SURF), 1300 away from the ND and 1.5 km underground, will measure the oscillated flux.

DUNE phase I far detector will have two Liquid Argon Time Projection Chambers (LArTPC), and is expected to operate by 2029. DUNE phase II will complement with two more modules by mid 30's, aiming at a total fiducial mass of 40 kilotonnes. The detector technology has been established, with excellent tracking and calorimetric capabilities, in several smaller-scale experiments \cite{icarus,microboone}. The first demonstration of the technology at the kilotonne scale has been carried out at the CERN Neutrino Platform as part of the ProtoDUNE program. In particular, the ProtoDUNE-SP experiment \cite{pdsp_tdr} reproduced the detector components of the second DUNE FD module \cite{dune_tdr4} at a scale 1:1 with a total argon mass 20 times smaller (770 tonnes). It was operated from mid-2018 to mid-2020, constituting the largest monolithic LArTPC to be built and operated up to date \cite{pdsp_1,pdsp_2}.

%Since neutrinos rarely interact, a dense target material is required. The relatively abundant noble-element argon, which liquified at 87 K has a density of 1.39 g/cm$^{3}$, is ideal as the active material of a TPC. However, The 3D images produced by charged particles traversing a LArTPC can be exponentially attenuated by medium impurities such as nitrogen, oxygen and water. This can introduce bias in energy measurements by absorbing ionization electrons. In ProtoDUNE-SP, the level of impurities was kept below 100 ppt oxygen equivalent \cite{pdsp_tdr} using a cryogenic recirculation and purification system. Three purity monitors, based on the ICARUS design \cite{PrMs}, were installed outside the active volume of the TPC to measure the electron lifetime, which is inversely related to the residual concentration of impurities. They ran twice a day to monitor the argon purity and thus to provide the necessary corrections for posterior data analysis. 

During ProtoDUNE-SP operation, the level of impurities was kept way below 100 ppt oxygen equivalent using a cryogenic recirculation and purification system~\cite{pdsp_tdr,pdsp_2}. This is fundamental in a LArTPC, as the 3D images produced by charged particles traversing the detector can be exponentially attenuated by medium impurities such as nitrogen, oxygen and water, biasing the energy measurements by absorbing ionization electrons. Three purity monitors, based on the ICARUS design \cite{PrMs}, were installed outside the active volume of the TPC to measure the electron lifetime, which is inversely related to the residual concentration of impurities. They ran twice a day to monitor the argon purity and thus to provide the necessary corrections for posterior data analysis. 

Achieving the necessary argon purity has been possible thanks to the studies based on computational fluid dynamic (CFD) simulations, and the experience gained with previous LArTPC demonstrators such as LAPD \cite{lapd} and the 35-tonne prototype \cite{35t_1,35t_2}, which have paved the way to operate large-scale cryostats requiring low concentrations of impurities. These studies have also shown that it exists a strong correlation between temperature and purity in the liquid argon volume. The distribution of impurities is insensitive to small ($\mathcal{O}\sim$1 K) absolute temperature variations, but strongly depends on the relative vertical temperature gradient. As the bulk volume in the cryostats needs to be continuously mixed with the incoming purified argon to ensure the purification of the whole LAr volume, the temperature distribution acts as a clear indicator of that mixing: a homogeneous temperature distribution indicates that the liquid is being properly mixed, whereas large temperature gradients constitute a clear symptom of deficient mixing. If the LAr bulk volume is not mixed properly, a stratification regime can develop: a significant portion of the liquid remains unpurified, generating `dead' regions inside the detector. Hence, a constant monitoring of this temperature gradient can identify and mitigate potential failures of the purification system. In ProtoDUNE-SP, this gradient was predicted to be about 15 mK by the CFD simulations \cite{pdsp_tdr,dune_tdr4}. 

The CFD simulations also predict, in spite of homogeneous mixing, that the concentration of impurities may also vary across the cryostat volume, requiring a position-dependent electron lifetime correction. The Purity Monitors themselves are intrusive objects which cannot be deployed inside the active volume, rather in only a few well-defined locations near the cryostat walls; thus, precise inference of the electron lifetime map requires alternative methods, as the one proposed in this article. A net of temperature sensors cross-calibrated to the $<$5 mK level should allow the measurement of a 15 mK temperature gradient, which can be used to constraint CFD simulations, providing a data-driven prediction of the impurity concentration. The main limitations of this new approach are the precision of the cross-calibration of the temperature sensors and the accuracy of the simulations. In this article, precise temperature monitoring for ProtoDUNE-SP will be described with particular emphasis on sensor calibration.

%Due to this, an alternative method to characterize the impurities concentration was proposed: using a net of temperature sensors relatively calibrated to characterize the gradients in the argon volume so that the CFD predictions about the impurities concentrations could be constrained. This approach presents a clear advantage with respect to the measurement with purity monitors: whereas the PrMs can only provide `point-like' measurements outside the active volume of the TPC, CFD simulations allow to infer the concentration of impurities anywhere inside the cryostat. The main limiting factors of this new approach are the precision of the cross-calibration of the temperature sensors and the accuracy of the simulations. In this article, precise temperature monitoring for the ProtoDUNE-SP experiment at CERN will be described, with special emphasis on the sensor calibration.